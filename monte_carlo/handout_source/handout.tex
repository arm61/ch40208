\documentclass[a4paper]{article}
\usepackage{titling}
\usepackage{authblk}
\usepackage{fancyhdr}
\usepackage{hyperref}
\usepackage{rsc}
\usepackage{siunitx}
\usepackage{graphicx}
\usepackage{listings}
\usepackage{color}
\usepackage{mhchem}

\definecolor{dkgreen}{rgb}{0,0.6,0}
\definecolor{gray}{rgb}{0.5,0.5,0.5}
\definecolor{mauve}{rgb}{0.58,0,0.82}

\lstset{frame=tb,
  language=Python,
  aboveskip=3mm,
  belowskip=3mm,
  showstringspaces=false,
  columns=flexible,
  basicstyle={\ttfamily},
  numbers=none,
  numberstyle=\tiny\color{gray},
  keywordstyle=\color{blue},
  commentstyle=\color{dkgreen},
  stringstyle=\color{mauve},
  breaklines=true,
  breakatwhitespace=true,
  tabsize=3
}
\DeclareSIUnit\Fahrenheit{\degree F}

\title{Exercise: Monte Carlo Methods in Action}
\author[1]{Dr Benjamin J. Morgan}
\author[1,2]{Dr Andrew R. McCluskey}
\affil[1]{Department of Chemistry, University of Bath, email: b.j.morgan@bath.ac.uk}
\affil[2]{Diamond Light Source, email: andrew.mccluskey@diamond.ac.uk}
\setcounter{Maxaffil}{0}
\renewcommand\Affilfont{\itshape\small}

\pagestyle{fancy}
\fancyhf{}
\rhead{CH40208}
\lhead{\thetitle}
\rfoot{\thepage}

\begin{document}
\maketitle

\section*{Aim}

In this exercise, you will develop Monte Carlo code for sampling the potential energy landspace of bonds stretchs and angle bends in a \ce{H_2O} molecule.

\section{Monte-Carlo sampling}

Monte-Carlo is a method for the sampling of some probability distribution.
We will introduce this sampling method by considering a group of children playing a game where they first draw a large circle and then a square around it, where the diameter of the circle ($d$) is equal to the length of the side of the square ($a$) on a beach (similar to that shown in Figure~\ref{fig:cir}).
The children then take turns randomly throwing pebbles into the square (with obviously some landing in the circle), with each throw counting as a \emph{trial} and each that lands inside the circle as a \emph{hit}.
%
\begin{figure}[t]
\centering
%\includegraphics{beach}
\label{fig:cir}
\caption{The pattern drawn on the beach.}
\end{figure}
%

The trials are technically sampling the area of the square ($A_s$), while the hits sample the area of the circle ($A_c$).
Therefore, the ratio of the number of hits ($N_{\text{hits}}$) to the number of trails ($N_{\text{trials}}$) is equal to the ratio of the area of the circle to the square,
%
\begin{equation}
  \frac{N_{\text{hits}}}{N_{\text{trials}}} = \frac{A_c}{A_s} = \frac{\pi(d/2)^2}{a^2} = \frac{\pi}{4}\text{, where }d=a.
\end{equation}

\section{Classical modelling}

\section{Potential energy landspace}


\bibliographystyle{rsc}
\bibliography{handout}

\end{document}
